\section{Task 2: Secure comparison between genomic data}
\subsection{Building Block: Estimating Set Union Cardinality}
Before describing our solution for Task 2, we first present the solution to estimate the size of union of two sets.
Suppose Alice and Bob have sets of elements $S^A = (e^A_1,...,e^A_n)$
and $S^B = (e^B_1,...,e^B_n)$ respectively. We want to find $|S^A\cup S^B|$. Here, we introduce two algorithms:

\paragraph{Using oblivious merge.}
A strawman approach of computing cardinality of the union is to use oblivious sorting, as detailed in Algorithm~\ref{alg1}.

\begin{algorithm}
\begin{algorithmic}[1]
\State Sort the input array $d[]$ obliviously.
\State $cnt = 1$
\For {$i=0:len(d)-1$}
	\If{$d[i] \neq d[i+1]$}
		\State $cnt = cnt + 1$
	\EndIf
\EndFor
\State\Return $cnt$
\end{algorithmic}
\caption{\textbf{Compute the size of union(d[]: a list of elements)}} %\\
\label{alg1}
\end{algorithm}
For $n$ elements, each with size $D$ bits, this approach requires a circuit of size $O(Dn\log^2n)$ using bitonic sorting network~\cite{bitonicsort},
and can be further reduced to $O(Dn\log n)$ if two parties sort their data locally and perform a bitonic merge using secure computation.
In the submission, we take the second approach to achieve a better performance.

The code for this approach is presented in Figure~\ref{fig:obl_merge}.
%\paragraph{Code used for oblivious merge.}
%Here we also include the code used of this algorithm in {\tt ObliVM-lang}.

\begin{figure}[H]
\begin{tabular}{rl}
\small 1&\small \tt \struct Task2Automated\at{m}\at{n}\{\};\\

\small 2&\small \tt int\at{n} Task2Automated\at{m}\at{n}.\func{funct}(int\at{m}[\public 1] key, \public int32 length) \{\\
\small 3&\small \tt \quad this.\func{obliviousMerge}(key, 0, length);\\
\small 4&\small \tt  \quad int\at{n} ret = 1;\\
\small 5&\small \tt  \quad \for(\public int32 i = 1; i < length; i = i + 1) \{\\
\small 6&\small \tt  \quad\quad     \ifs(key[i-1] != key[i])\\
\small 7&\small \tt        \quad\quad\quad  ret = ret + 1;\\
\small 8&\small \tt    \quad\}\\
\small 9&\small \tt    \quad \return ret;\\
\small 10&\small \tt \}\\
\small 11&\small \tt void Task2Automated\at{m}\at{n}.\func{obliviousMerge}(int\at{m}[\public 1] key, \public int32 lo, \public int32 l) \{\\
\small 12&\small \tt   \quad \ifs (l > 1) \{\\
\small 13&\small \tt     \quad\quad  \public int32 k = 1;\\
\small 14&\small \tt    \quad\quad   \while (k < l) k = k << 1;\\
\small 15&\small \tt     \quad\quad\quad  k = k >> 1;\\
\small 16&\small \tt    \quad\quad\quad   \for (\public int32 i = lo; i < lo + l - k; i = i + 1)\\
\small 17&\small \tt     \quad\quad\quad\quad     this.\func{compare}(key, i, i + k);\\
\small 18&\small \tt     \quad\quad\quad  this.\func{obliviousMerge}(key, lo, k);\\
\small 19&\small \tt     \quad\quad\quad  this.\func{obliviousMerge}(key, lo + k, l - k);\\
\small 20&\small \tt   \quad\}\\
\small 21&\small \tt \}\\


\small 22&\small \tt void Task2Automated\at{m}\at{n}.\func{compare}(int\at{m}[\public 1] key, \public int32 i, \public int32 j) \{\\
\small 23&\small \tt  \quad  int\at{m} tmp = key[j];\\
\small 24&\small \tt  \quad  int\at{m} tmp2 = key[i];\\
\small 25&\small \tt \quad   \ifs( key[i] < key[j] )\\
\small 26&\small \tt   \quad\quad    tmp = key[i];\\
\small 27&\small \tt  \quad  tmp = tmp $\wedge$ key[i];\\
\small 28&\small \tt \quad   key[i] = tmp $\wedge$ key[j];\\
\small 29&\small \tt \quad   key[j] = tmp $\wedge$ tmp2;\\
\small 30&\small \tt \}\\
\end{tabular}
\caption{Code for oblivious merge written in {\tt ObliVM}}
\label{fig:obl_merge}
\end{figure}


\paragraph{Using bloom filter.}
It is known that bloom filters can be used to check the existence of an element in a set. However, bloom filters can also be used
to estimate the capacity of a set. Let $X$ be the number of bits set, $m$ be the total number of bits used in the bloom filter and
$k$ be the number of hash functions used. Number of elements in
%size of a 
the bloom filter can be estimated as 
$$-\frac{m\ln(1-\frac{X}{m})}{k}.$$
So, in order to compute the union of two sets, each party first builds their own bloom filter locally using the same set of hash functions.
Then, in secure computation, the two parties union the bloom filter using a bitwise OR, and count number of ones in the new bit array
($X$ mentioned above).
Note that after getting X, the remaining part of the computation can  be done in cleartext.

The code for this approach is presented in Figure~\ref{fig:bf_merge}.

\begin{figure}[H]
\begin{tabular}{rl}
\small 1&\small \tt \struct Pair<T1, T2> \{\\
\small 2&\small \tt  \quad  T1 left;\\
\small 3&\small \tt  \quad  T2 right;\\
\small 4&\small \tt \};\\
\small 5&\small \tt \struct bit \{\\
\small 6&\small \tt  \quad  int1 v;\\
\small 7&\small \tt \};\\

\small 8&\small \tt \struct Int\at{n} \{\\
\small 9&\small \tt  \quad  int\at{n} v;\\
\small 10&\small \tt \};\\
\small 11&\small \tt \struct BF\_circuit\{\};\\

\small 12&\small \tt Pair<bit, Int\at{n}> BF\_circuit.\func{add}\at{n}(int\at{n} x, int\at{n} y) \{\\
\small 13&\small \tt \quad   bit cin;\\
\small 14&\small \tt  \quad  Int\at{n} ret;\\
\small 15&\small \tt  \quad  bit t1, t2;\\
\small 16&\small \tt  \quad  \for(\public int32 i=0; i<n; i = i+1) \{\\
\small 17&\small \tt   \quad\quad    t1.v = x\$i\$  $\wedge$ cin.v;\\
\small 18&\small \tt    \quad\quad   t2.v = y\$i\$  $\wedge$ cin.v;\\
\small 19&\small \tt   \quad\quad    ret.v\$i\$ = x\$i\$  $\wedge$ t2.v;\\
\small 20&\small \tt    \quad\quad   t1.v = t1.v \& t2.v;\\
\small 21&\small \tt     \quad\quad  cin.v = cin.v  $\wedge$ t1.v;\\
\small 22&\small \tt   \quad \}\\
\small 23&\small \tt   \quad \return Pair\{bit, Int\at{n}\}(cin, ret);\\
\small 24&\small \tt \}\\

\small 25&\small \tt int\at{log(n+1)} BF\_circuit.\func{countOnes}\at{n}(int\at{n} x) \{\\
\small 26&\small \tt   \quad \ifs(n==1) \return x;\\
\small 27&\small \tt   \quad int\at{log(n-n/2+1)} first = this.\func{countOnes}\at{(n/2)}(x\$0\~{}n/2\$);\\
\small 28&\small \tt    \quad int\at{log(n-n/2+1)} second = this.\func{countOnes}\at{(n-n/2)}(x\$n/2\~{}n\$);\\
\small 29&\small \tt \quad   Pair<bit, Int\at{log(n-n/2)}> ret = this.\func{add}\at{log(n-n/2+1)}(first, second);\\

\small 30&\small \tt \quad   int\at{log(n+1)} r = ret.right.v;\\
\small 31&\small \tt \quad   r\$log(n+1)-1\$ = ret.left.v;\\
\small 32&\small \tt \quad   \return r;\\
\small 33&\small \tt \}\\
\small 34&\small \tt int\at{log(n+1)} BF\_circuit.\func{merge}\at{n}(int\at{n} x, int\at{n} y) \{\\
\small 35&\small \tt  \quad  int\at{n} tmp;\\
\small 36&\small \tt  \quad  \for(\public int32 i = 0; i < n; i = i +1 ) \{\\
\small 37&\small \tt    \quad\quad   tmp\$i\$ = x\$i\$ | y\$i\$;\\
\small 38&\small \tt   \quad \}\\
\small 39&\small \tt    \quad\return this.\func{countOnes}\at{n}(tmp);\\
\small 40&\small \tt \}\\
\end{tabular}
\caption{Code for bloom filter approach written in {\tt ObliVM}}
\label{fig:bf_merge}
\end{figure}

\subsection{Task 2a, Hamming Distance}
\paragraph{Problem statement.}
In this problem, we want to compute the hamming distance defined on the website:
\begin{framed}
{\tt~\\
d = 0;\\
for all records in the VCF files, which have SVTYPE = SNP or SUB: if given a chrom and pos, there is only one record in one of the VCF file (e.g., x != null), then we set the other record as NULL (e.g., y == null)\\
if (x == null) || (y == null) || (x.ref == y.ref \&\& x.alt != y.alt)

   d += 1;\\
end for\\
}
\end{framed}
Alice and Bob each holds a list of records, where each record is of the format $(pos, val)$.

\paragraph{Our solution.}
Each party constructs a set containing all the records from the input: $S^A$ for Alice and $S^B$ for Bob.
Hamming distance defined above is equivalent to $|S^A\cup S^B| - |S^A\cap S^B|$, that is the sum of number of elements not shared by two parties.
Note that $|S^A\cup S^B| - |S^A\cap S^B| = 2\times|S^A\cup S^B|-|S^A| - |S^B|$. So we can use the aforementioned algorithms to compute hamming distance.

Note that in order to do oblivious merge, each record has to be of the same bitlength. Instead of padding every record to the maximum possible length, we hash each
record to a fixed length bit string. In the code, we hash it to 64-bit numbers, which gives a failure probability of $ $.

\subsection{Task2b, Edit Distance}
\paragraph{Problem statement.}
In this problem, we want to compute the edit distance defined on the website:
\begin{framed}
{\tt~\\
d = 0;\\
for all records in the VCF files:\\
1. if x == y, continue;\\
2. if x != y, d += max(D(x), D(y))\\
end for\\
where D(x):\\
if x.svtype == snp, D(x) = 1\\
if x.svtype == sub, D(x) = len(x)\\
if x.svtype == ins, D(x) = len(x)\\
if x.svtype == del, D(x) = len(x)\\
}\end{framed}

The input to this problem is the same as Task2a.
\paragraph{Our solution.}
We first compute {\tt d1} defined as follows:
\begin{framed}
{\tt~\\
d1 = 0;\\
for every pos in VCF files:\\
if there are two records x, y at pos,

d1 += max(D(x), D(y))\\
else if there is only one record at pos,

d1+=D(x)\\}
\end{framed}

In order to compute {\tt d1}, each party construct a new set as follows:  for every record $(pos, val)$,
each party insert $(pos, i), i\in[1, len(val)]$ to a new set and get set $S_1^A, S_1^B$ for Alice and Bob. Then we compute
${\tt d1} = |S_1^A\cup S_1^B|$.

Then we compute ${\tt d2}$ defined as follows:
\begin{framed}
{\tt~\\
d2 = 0;\\
for every pos in VCF files:\\
if there are two records x, y at pos and they are same,

d2 += D(x)\\}
\end{framed}
In order to compute {\tt d1}, each party construct a new set as follows:  for every record $(pos, val, i)$,
each party insert $(pos, val, i), i\in[1, len(val)]$ to a new set and get set $S_2^A, S_2^B$ for Alice and Bob. Then we compute
${\tt d2} = |S_2^A\cup S_2^B|$.


Finally we compute {\tt d = d2 - d1}.