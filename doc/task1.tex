\section{Task 1: Secure Distributed GWAS}
For all tasks, we will briefly 
repeat the problem statement.
For a more detailed description of their application
contexts, we refer the readers to the
competition website~\cite{idash}.

\subsection*{Task 1a: Computing Minor Allele Frequencies}
\paragraph{Problem statement.}
~\\
Input for Alice: List of alleles $l^A = (e^A_1,...,e^A_n)$ 
where each $e^A_i$ is from $\{A, T, C, G\}$.\\
Input for Bob: List of alleles $l^B = (e^B_1,...,e^B_n)$
where each $e^B_i$ is from $\{A, T, C, G\}$.\\

The problem is to compute the 
Minor Allele Frequencies(MAF) over 
the combined input $l = l^A || l^B$. 
The definition of MAF is as follows:
Let $f_1$ and $f_2$ be the frequency of two types of alleles in the
combined input $l$ (it is guaranteed
that only two types of alleles will appear). 
MAF is defined to be as $min\{f_1, f_2\}$.

\paragraph{Our solution.}
The two parties first compute the frequency of two types of alleles of their own lists and get
$(f^A_1, f^A_2)$ for Alice and $(f^B_1,f^B_2)$ for Bob, ordered by allele type.
Then, over a secure computation protocol, 
the two parties first sum up the frequency by
$$(f_1, f_2) = (f^A_1+f^B_1, f^A_2+f^B_2),$$
and then report the smaller number between $f_1$ and $f_2$.

The code used for this task is shown in Figure~\ref{fig:lang-circuit-oram}.
\begin{figure}[H]
\begin{tabular}{rl}
\small 1&\small \tt	\struct Task1aAutomated\at{m}\at{n}\{\};\\
\small 2&\small \tt	void Task1aAutomated\at{m}\at{n}.\func{funct}(int\at{m}[\public n] alice\_data,\\
\small 2&\small \tt \quad \quad \qquad \qquad \qquad \qquad\qquad\qquad\qquad int\at{m}[\public n] bob\_data,\\
\small 3&\small \tt	\quad \quad \qquad \qquad \qquad \qquad\qquad\qquad\qquad    int\at{m}[\public n] ret,\\
\small 3&\small \tt	\quad \quad \qquad \qquad \qquad \qquad\qquad\qquad\qquad \public int\at{m} total\_instances) \{\\
\small 4&\small \tt	\quad   int\at{m} total = total\_instances;\\
\small 5&\small \tt	  \quad int\at{m} half = total\_instances / 2;\\
\small 6&\small \tt	 \quad  \for(\public int32 i = 0; i < n; i = i + 1) \{\\
\small 7&\small \tt	     \quad\quad ret[i] = alice\_data[i] + bob\_data[i];\\
\small 8&\small \tt	     \quad\quad \ifs(ret[i] > half)\\
\small 9&\small \tt	         \quad\quad\quad ret[i] = total - ret[i];\\
\small 10&\small \tt	  \quad \}\\
\small 11&\small \tt\}\\
\end{tabular}
\caption{Code for Task 1a written in {\tt ObliVM-lang}}
\label{fig:lang-circuit-oram}
\end{figure}


\paragraph{Results.}
For this task, each test case requires only 40 AND gates for both manually generated circuits and automatically generated circuits.

%\paragraph{Code used for Task1a.}
%Here we also include the code used for Task 1a written in {\tt ObliVM-lang}.

\subsection*{Task 1b: Computing $\chi^2$ statistics}
\paragraph{Problem statement.}
~\\
Input for Alice: two lists of alleles $l^A_{case} = (e^A_1,...,e^A_n)$, $l^A_{control} = (e'^A_1,...,e'^A_n)$ where each $e^A_i, e'^A_i$ is from $\{A, T, C, G\}$.\\
Input for Bob: two lists of alleles $l^B_{case} = (e^B_1,...,e^B_n)$, $l^B_{control} = (e'^B_1,...,e'^B_n)$ where each $e^B_i, e'^B_i$ is from $\{A, T, C, G\}$.\\

First let us define $a, b$ to be the frequency of two types of alleles in
$l_{case} = l^A_{case} || l^B_{case}$ and let $c, d$ be the frequency of two types of alleles in
$l_{control} = l^A_{control} || l^B_{control}$. 
(it is also guaranteed
that only two types of alleles will appear). 
The problem is to compute the $\chi^2$ statistics over $l_{case}$ and $l_{control}$, defined as:
$$n\times\frac{(ad-bc)^2}{rsgk},$$
where $r = a + b, s = c + d, g = a + c, k = b + d, n =  r + s$; 


\paragraph{Our solution.}
Each party first locally computes the frequency of alleles on their own list: $a^A,b^A,c^A,d^A$ for Alice
and $a^B,b^B,c^B,d^B$ for Bob. $a^A,b^A$ are the frequency of two types of alleles in $l^A_{case}$ and 
$c^A,d^A$ are the frequency of two types of alleles in $l^A_{control}$. $a^B,b^B,c^B,d^B$  is defined similarly.
In the secure computation we first compute $a = a^A + a^B, b = b^A + b^B,
c = c^A + c^B,d = d^A + d^B$, then we evaluate the equation mentioned above directly using floating point
numbers.

The code used for this task is shown in Figure~\ref{fig:task1b}.
%\paragraph{Code used for Task1b.}
%Here we also include the code used for Task1b written in {\tt ObliVM-lang}.
\begin{figure}[H]
\begin{tabular}{rl}
\small 1&\small \tt \struct Task1bAutomated\at{n}\{\};\\
\small 2&\small \tt float32[\public n] Task1bAutomated\at{n}.\func{func}(\\
\small 3&\small \tt \quad \quad     float32[\public n][\public 2] alice\_case, float32[\public n][\public 2] alice\_control,\\
\small 4&\small \tt  \quad \quad    float32[\public n][\public 2] bob\_case, float32[\public n][\public 2] bob\_control) \{\\
\small 5&\small \tt \quad   float32[\public n] ret;\\
\small 6&\small \tt  \quad  \for(\public int32 i = 0; i < n; i = i + 1) \{\\
\small 7&\small \tt     \quad \quad  float32 a = alice\_case[i][0] + bob\_case[i][0];\\
\small 8&\small \tt     \quad \quad  float32 b = alice\_case[i][1] + bob\_case[i][1];\\
\small 9&\small \tt     \quad \quad  float32 c = alice\_control[i][0] + bob\_control[i][0];\\
\small 10&\small \tt    \quad \quad   float32 d = alice\_control[i][1] + bob\_control[i][1];\\
\small 11&\small \tt    \quad \quad   float32 g = a + c, k = b + d;\\
\small 12&\small \tt    \quad \quad   float32 tmp = a*d - b*c;\\
\small 13&\small \tt     \quad \quad  tmp = tmp*tmp;\\
\small 14&\small \tt     \quad \quad  ret[i] = tmp / (g * k);\\
\small 15&\small \tt   \quad  \}\\
\small 16&\small \tt  \quad  \return ret;\\
\small 17&\small \tt\}\\
\end{tabular}
\caption{Code for Task 1b written in {\tt ObliVM-lang}}
\label{fig:task1b}
\end{figure}
\paragraph{Results.}
Our implementation for this task supports an arbitrary 
trade-off between precision and speed. We mention two specific cases here.
For each test case, an implementation that requires 7763 AND gates achieves a maximum absolute error of $1.11\times10^{-4}$
 and an implementation with 14443 AND gates achieves a maximum absolute error of $5.6\times10^{-8}$.


%%% Local Variables: 
%%% mode: latex
%%% TeX-master: "main"
%%% End: 
