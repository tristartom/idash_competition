\documentclass[11pt]{article}
\usepackage{algorithm}
\usepackage{algpseudocode}
\usepackage{algorithmicx, color, xspace, xcolor, fullpage, url}
\usepackage{framed}

\definecolor{brown}{RGB}{139, 137, 137}
\definecolor{maroon}{RGB}{178, 34, 34}
\definecolor{green}{RGB}{34, 139, 34}
\definecolor{types}{RGB}{72, 61, 139}

\newcommand{\intn}[1]{\ensuremath{\tt {\color{brown} int#1\xspace}}}
\newcommand{\rndn}[1]{\ensuremath{\tt {\color{brown} rnd#1\xspace}}}
\newcommand{\struct}[0]{\ensuremath{\tt {\color{blue} struct~}}}
\newcommand{\public}[0]{\ensuremath{\tt {\color{maroon} public~}}}
\newcommand{\secure}[0]{\ensuremath{\tt {\color{maroon} secure~}}}
\newcommand{\at}[1]{\ensuremath{\tt {\color{green} @#1\xspace}}}
\newcommand{\bit}[1]{\ensuremath{\tt {\color{green} #1\xspace}}}
\newcommand{\types}[1]{{\tt {\color{types} <#1>}}}
\newcommand{\type}[1]{{\tt {\color{types} #1}}}
\newcommand{\while}[0]{\ensuremath{\tt {\color{blue} while~}}}
\newcommand{\bound}[1]{\ensuremath{\tt {\color{blue} bfor}(#1)}}
\newcommand{\for}[0]{\ensuremath{\tt {\color{blue} for~}}}
\newcommand{\ifs}[0]{\ensuremath{\tt {\color{blue} if~}}}
\newcommand{\then}[0]{\ensuremath{\tt {\color{blue} then~}}}
\newcommand{\elses}[0]{\ensuremath{\tt {\color{blue} else~}}}
\newcommand{\return}[0]{\ensuremath{\tt {\color{blue} return~}}}
\newcommand{\define}[0]{\ensuremath{\tt {\color{blue} define~}}}
\newcommand{\typedef}[0]{\ensuremath{\tt {\color{blue} typedef~}}}
\newcommand{\native}[0]{\ensuremath{\tt {\color{blue} native~}}}
\newcommand{\partf}[0]{\ensuremath{\tt {\color{blue} phantom~}}}
\newcommand{\dummy}[0]{\ensuremath{\tt {\color{blue} dummy~}}}
\newcommand{\func}[1]{\ensuremath{\tt {\color{orange} #1}}}



\title{iDASH Secure Genome Analysis Competition Using {\sf ObliVM}}
\author{Xiao Shaun Wang, Chang Liu, Kartik Nayak, Justin Wagner, Yan Huang and Elaine Shi}
\date{University of Maryland, College Park\\
{\tt \{wangxiao,liuchang,kartik,jwagner,elaine\}@cs.umd.edu, yh33@indiana.edu}}
\begin{document}
\maketitle
\section{Overview}
We implemented all the problems using the secure computation framework {\sf ObliVM}~\footnote{\url{http://www.oblivm.com}}~\cite{oblivm}.
For each question, we provide multiple solutions, with manually built circuits and automatically built circuits:
\begin{enumerate}
\item Task1a: we provide two solutions, one built manually, one built using compiler. Both of them achieve exact result with no error.
\item Task1b: we provide three solutions, two built manually, one built using compiler. The second manually built circuit can achieve a trade-off between efficiency and accuracy.
\item Task2a/b: we provide four solutions, one built manually using oblivious merge, one built with compiler using oblivious merge, one built manually using bloom filter, and one built using compiler. The one with bloom filter can be tuned for a trade-off between efficiency and accuracy.
\end{enumerate}

All the implementations will be open sourced at \url{https://github.com/wangxiao1254/idash_competition} after the deadline of the competition.
\section{Task1}
\subsection{Task1a, Computing Minor Allele Frequencies}
The problem of computing Minor Allele Frequencies(MAF) can be abstracted as follows:

Suppose Alice and Bob each has a list of alleles $l^a = (e^a_1,...,e^a_n)$ and $l^b = (e^b_1,...,e^b_n)$.
Let's append $l^b$ to $l^a$ and get $l = l^a || l^b$. It is known that in $l$, there will be at most two types of
alleles from $(A,T,C,G)$. We want to compute the frequency of allele in $l$ that appears less frequently.

\paragraph{Our Solution}
The two parties first aggregate their own input into two numbers $(f^a_1, f^a_2)$ for Alice and $(f^b_1,f^b_2)$ for Bob, ordered by allele type.
Then, in the secure computation, the two parties first aggregate the frequency by
$$(f_1, f_2) = (f^a_1+f^a_1, f^b_2+f^b_2),$$
and then report the smaller number between $f_1$ and $f_2$

For this task, each test case only requires 40 AND gates for both manually generated circuits and automatically generated circuits.

\paragraph{Code used for Task1a.}
Here we also include the code used for Task1a in {\tt ObliVM-lang}.
\begin{figure}[H]
\begin{tabular}{rl}
\small 1&\small \tt	\struct Task1aAutomated\at{m}\{\};\\
\small 2&\small \tt	void Task1aAutomated\at{m}.funct(int\at{m}[\public 1] alice\_data, int\at{m}[\public 1] bob\_data,\\
\small 3&\small \tt	\quad      int\at{m}[\public 1] ret, \public int\at{m} total\_instances, \public int32 test\_cases) \{\\
\small 4&\small \tt	\quad   int\at{m} total = total\_instances;\\
\small 5&\small \tt	  \quad int\at{m} half = total\_instances / 2;\\
\small 6&\small \tt	 \quad  \for(\public int32 i = 0; i < test\_cases; i = i + 1) \{\\
\small 7&\small \tt	     \quad\quad ret[i] = alice\_data[i] + bob\_data[i];\\
\small 8&\small \tt	     \quad\quad \ifs(ret[i] > half)\\
\small 9&\small \tt	         \quad\quad\quad ret[i] = total - ret[i];\\
\small 10&\small \tt	  \quad \}\\
\small 11&\small \tt\}\\
\end{tabular}
\label{fig:lang-circuit-oram}
\end{figure}


\subsection{Task1b, Computing $\chi$ square statistics}
The problem can be abstracted as follows:
The two parties want to compute the following results:
$$n\times\frac{(ad-bc)^2}{rsgk},$$
where $r = a + b, s = c + d, g = a + c, k = b + d, n =  r + s$; and $a,b,c,d$ are additively secret shared by two parties.

\paragraph{Our Solution}
In the high level, our solution folows the direct approach: in secure computation, we first add shares from two parties and get $a,b,c,d$
and then convert it into floating point numbers securely, and finally compute the function mentioned above directly.

For this task, each test case only requires 7763 AND gates achieving maximum absolute error of $1.11\times0^{-4}$
and 14443 AND gates achieving maximum absolute error of $5.6\times0^{-8}$.

\paragraph{Code used for Task1b.}
Here we also include the code used for Task1b in {\tt ObliVM-lang}.
\begin{figure}[H]
\begin{tabular}{rl}
\small 1&\small \tt \struct Task1bAutomated\at{n}\{\};\\
\small 2&\small \tt float32[\public n] Task1bAutomated\at{n}.func(\\
\small 3&\small \tt \quad      float32[\public n][\public 3] alice\_case, float32[\public n][\public 3] alice\_control,\\
\small 4&\small \tt  \quad     float32[\public n][\public 3] bob\_case, float32[\public n][\public 3] bob\_control) \{\\
\small 5&\small \tt \quad   float32[\public n] ret;\\
\small 6&\small \tt  \quad  \for(\public int32 i = 0; i < n; i = i + 1) \{\\
\small 7&\small \tt     \quad \quad  float32 a = alice\_case[i][0] + bob\_case[i][0];\\
\small 8&\small \tt     \quad \quad  float32 b = alice\_case[i][1] + bob\_case[i][1];\\
\small 9&\small \tt     \quad \quad  float32 c = alice\_control[i][0] + bob\_control[i][0];\\
\small 10&\small \tt    \quad \quad   float32 d = alice\_control[i][1] + bob\_control[i][1];\\
\small 11&\small \tt    \quad \quad   float32 g = a + c, k = b + d;\\
\small 12&\small \tt    \quad \quad   float32 tmp = a*d - b*c;\\
\small 13&\small \tt     \quad \quad  tmp = tmp*tmp;\\
\small 14&\small \tt     \quad \quad  ret[i] = tmp / (g * k);\\
\small 15&\small \tt   \quad  \}\\
\small 16&\small \tt  \quad  \return ret;\\
\small 17&\small \tt\}\\
\end{tabular}
\end{figure}

\section{Task 2}
\subsection{Building Block: Estimating Set Union Cardinality}
Before describing our solution for Task 2, we first present the solution to estimate the size of union of two sets.
Suppose Alice and Bob have sets of elements $S^A = (e^A_1,...,e^A_n)$
and $S^B = (e^B_1,...,e^B_n)$ respectively. We want to find $|S^A\cup S^B|$. Here, we introduce two algorithms:

\paragraph{Using oblivious merge.}
A strawman approach of computing cardinality of the union is to use oblivious sorting, as detailed in Algorithm~\ref{alg1}.

\begin{algorithm}
\begin{algorithmic}[1]
\State Sort the input array $d[]$ obliviously.
\State $cnt = 1$
\For {$i=0:len(d)-1$}
	\If{$d[i] \neq d[i+1]$}
		\State $cnt = cnt + 1$
	\EndIf
\EndFor
\State\Return $cnt$
\end{algorithmic}
\caption{\textbf{Compute size of union}} %\\
\label{alg1}
\end{algorithm}
For $n$ elements, each with size $D$ bits, this approach requires a circuit of size $O(Dn\log^2n)$ using bitonic sorting network~\cite{bitonicsort},
and can be further reduced to $O(Dn\log n)$ if two parties sort their data locally and perform a bitonic merge using secure computation.
In the submission, we take the second approach to achieve a better performance.

The code for this approach is presented in Figure~\ref{fig:obl_merge}.
%\paragraph{Code used for oblivious merge.}
%Here we also include the code used of this algorithm in {\tt ObliVM-lang}.

\begin{figure}[H]
\begin{tabular}{rl}
\small 1&\small \tt \struct Task2Automated\at{m}\at{n}\{\};\\

\small 2&\small \tt int\at{n} Task2Automated\at{m}\at{n}.\func{funct}(int\at{m}[\public 1] key, \public int32 length) \{\\
\small 3&\small \tt \quad this.\func{obliviousMerge}(key, 0, length);\\
\small 4&\small \tt  \quad int\at{n} ret = 1;\\
\small 5&\small \tt  \quad \for(\public int32 i = 1; i < length; i = i + 1) \{\\
\small 6&\small \tt  \quad\quad     \ifs(key[i-1] != key[i])\\
\small 7&\small \tt        \quad\quad\quad  ret = ret + 1;\\
\small 8&\small \tt    \quad\}\\
\small 9&\small \tt    \quad \return ret;\\
\small 10&\small \tt \}\\
\small 11&\small \tt void Task2Automated\at{m}\at{n}.\func{obliviousMerge}(int\at{m}[\public 1] key, \public int32 lo, \public int32 l) \{\\
\small 12&\small \tt   \quad \ifs (l > 1) \{\\
\small 13&\small \tt     \quad\quad  \public int32 k = 1;\\
\small 14&\small \tt    \quad\quad   \while (k < l) k = k << 1;\\
\small 15&\small \tt     \quad\quad\quad  k = k >> 1;\\
\small 16&\small \tt    \quad\quad\quad   \for (\public int32 i = lo; i < lo + l - k; i = i + 1)\\
\small 17&\small \tt     \quad\quad\quad\quad     this.\func{compare}(key, i, i + k);\\
\small 18&\small \tt     \quad\quad\quad  this.\func{obliviousMerge}(key, lo, k);\\
\small 19&\small \tt     \quad\quad\quad  this.\func{obliviousMerge}(key, lo + k, l - k);\\
\small 20&\small \tt   \quad\}\\
\small 21&\small \tt \}\\


\small 22&\small \tt void Task2Automated\at{m}\at{n}.\func{compare}(int\at{m}[\public 1] key, \public int32 i, \public int32 j) \{\\
\small 23&\small \tt  \quad  int\at{m} tmp = key[j];\\
\small 24&\small \tt  \quad  int\at{m} tmp2 = key[i];\\
\small 25&\small \tt \quad   \ifs( key[i] < key[j] )\\
\small 26&\small \tt   \quad\quad    tmp = key[i];\\
\small 27&\small \tt  \quad  tmp = tmp $\wedge$ key[i];\\
\small 28&\small \tt \quad   key[i] = tmp $\wedge$ key[j];\\
\small 29&\small \tt \quad   key[j] = tmp $\wedge$ tmp2;\\
\small 30&\small \tt \}\\
\end{tabular}
\caption{Code for oblivious merge written in {\tt ObliVM}}
\label{fig:obl_merge}
\end{figure}


\paragraph{Using Bloom Filter.}
It is known that bloom filters can be used to check the existence of an element in a set. However, bloom filters can also be used
to estimate the capacity of a set. Let $X$ be the number of bits set, $m$ be the total number of bits used in the bloom filter and
$k$ be the number of hash functions used. Number of elements in
%size of a 
the bloom filter can be estimated as 
$$-\frac{m\ln(1-\frac{X}{m})}{k}.$$
So, in order to compute the union of two sets, each party first builds their own bloom filter locally using the same set of hash functions.
Then, in secure computation, the two parties union the bloom filter using a bitwise OR, and count number of ones in the new bit array
($X$ mentioned above).
Note that after getting X, the remaining part of the computation can  be done in cleartext.

The code for this approach is presented in Figure~\ref{fig:bf_merge}.

\begin{figure}[H]
\begin{tabular}{rl}
\small 1&\small \tt \struct Pair<T1, T2> \{\\
\small 2&\small \tt  \quad  T1 left;\\
\small 3&\small \tt  \quad  T2 right;\\
\small 4&\small \tt \};\\
\small 5&\small \tt \struct bit \{\\
\small 6&\small \tt  \quad  int1 v;\\
\small 7&\small \tt \};\\

\small 8&\small \tt \struct Int\at{n} \{\\
\small 9&\small \tt  \quad  int\at{n} v;\\
\small 10&\small \tt \};\\
\small 11&\small \tt \struct BF\_circuit\{\};\\

\small 12&\small \tt Pair<bit, Int\at{n}> BF\_circuit.\func{add}\at{n}(int\at{n} x, int\at{n} y) \{\\
\small 13&\small \tt \quad   bit cin;\\
\small 14&\small \tt  \quad  Int\at{n} ret;\\
\small 15&\small \tt  \quad  bit t1, t2;\\
\small 16&\small \tt  \quad  \for(\public int32 i=0; i<n; i = i+1) \{\\
\small 17&\small \tt   \quad\quad    t1.v = x\$i\$  $\wedge$ cin.v;\\
\small 18&\small \tt    \quad\quad   t2.v = y\$i\$  $\wedge$ cin.v;\\
\small 19&\small \tt   \quad\quad    ret.v\$i\$ = x\$i\$  $\wedge$ t2.v;\\
\small 20&\small \tt    \quad\quad   t1.v = t1.v \& t2.v;\\
\small 21&\small \tt     \quad\quad  cin.v = cin.v  $\wedge$ t1.v;\\
\small 22&\small \tt   \quad \}\\
\small 23&\small \tt   \quad \return Pair\{bit, Int\at{n}\}(cin, ret);\\
\small 24&\small \tt \}\\

\small 25&\small \tt int\at{log(n+1)} BF\_circuit.\func{countOnes}\at{n}(int\at{n} x) \{\\
\small 26&\small \tt   \quad \ifs(n==1) \return x;\\
\small 27&\small \tt   \quad int\at{log(n-n/2+1)} first = this.\func{countOnes}\at{(n/2)}(x\$0\~{}n/2\$);\\
\small 28&\small \tt    \quad int\at{log(n-n/2+1)} second = this.\func{countOnes}\at{(n-n/2)}(x\$n/2\~{}n\$);\\
\small 29&\small \tt \quad   Pair<bit, Int\at{log(n-n/2)}> ret = this.\func{add}\at{log(n-n/2+1)}(first, second);\\

\small 30&\small \tt \quad   int\at{log(n+1)} r = ret.right.v;\\
\small 31&\small \tt \quad   r\$log(n+1)-1\$ = ret.left.v;\\
\small 32&\small \tt \quad   \return r;\\
\small 33&\small \tt \}\\
\small 34&\small \tt int\at{log(n+1)} BF\_circuit.\func{merge}\at{n}(int\at{n} x, int\at{n} y) \{\\
\small 35&\small \tt  \quad  int\at{n} tmp;\\
\small 36&\small \tt  \quad  \for(\public int32 i = 0; i < n; i = i +1 ) \{\\
\small 37&\small \tt    \quad\quad   tmp\$i\$ = x\$i\$ | y\$i\$;\\
\small 38&\small \tt   \quad \}\\
\small 39&\small \tt    \quad\return this.\func{countOnes}\at{n}(tmp);\\
\small 40&\small \tt \}\\
\end{tabular}
\caption{Code for bloom filter approach written in {\tt ObliVM}}
\label{fig:bf_merge}
\end{figure}

\subsection{Task 2a: Hamming Distance}
\paragraph{Problem statement.}
In this problem, we want to compute the hamming distance defined on the website:
\begin{framed}
{\tt~\\
d = 0;\\
for all records in the VCF files, which have SVTYPE = SNP or SUB: if given a chrom and pos, there is only one record in one of the VCF file (e.g., x != null), then we set the other record as NULL (e.g., y == null)\\
if (x == null) || (y == null) || (x.ref == y.ref \&\& x.alt != y.alt)

   d += 1;\\
end for\\
}
\end{framed}
Alice and Bob each holds a list of records, where each record is of the format $(pos, val)$.

\paragraph{Our solution.}
Each party constructs a set containing all the records from the input: $S^A$ for Alice and $S^B$ for Bob.
Hamming distance defined above is equivalent to $|S^A\cup S^B| - |S^A\cap S^B|$, that is the sum of number of elements not shared by two parties.
Note that $|S^A\cup S^B| - |S^A\cap S^B| = 2\times|S^A\cup S^B|-|S^A| - |S^B|$. So we can use the aforementioned algorithms to compute hamming distance.

Note that in order to do oblivious merge, each record has to be of the same bitlength. Instead of padding every record to the maximum possible length, we hash each
record to a fixed length bit string. In the code, we hash it to 64-bit numbers, which gives a failure probability of $ $.

\subsection{Task2b, Edit Distance}
\paragraph{Problem statement.}
In this problem, we want to compute the edit distance defined on the website:
\begin{framed}
{\tt~\\
d = 0;\\
for all records in the VCF files:\\
1. if x == y, continue;\\
2. if x != y, d += max(D(x), D(y))\\
end for\\
where D(x):\\
if x.svtype == snp, D(x) = 1\\
if x.svtype == sub, D(x) = len(x)\\
if x.svtype == ins, D(x) = len(x)\\
if x.svtype == del, D(x) = len(x)\\
}\end{framed}

The input to this problem is the same as Task2a.
\paragraph{Our solution.}
We first compute {\tt d1} defined as follows:
\begin{framed}
{\tt~\\
d1 = 0;\\
for every pos in VCF files:\\
if there are two records x, y at pos,

d1 += max(D(x), D(y))\\
else if there is only one record at pos,

d1+=D(x)\\}
\end{framed}

In order to compute {\tt d1}, each party construct a new set as follows:  for every record $(pos, val)$,
each party insert $(pos, i), i\in[1, len(val)]$ to a new set and get set $S_1^A, S_1^B$ for Alice and Bob. Then we compute
${\tt d1} = |S_1^A\cup S_1^B|$.

Then we compute ${\tt d2}$ defined as follows:
\begin{framed}
{\tt~\\
d2 = 0;\\
for every pos in VCF files:\\
if there are two records x, y at pos and they are same,

d2 += D(x)\\}
\end{framed}
In order to compute {\tt d1}, each party construct a new set as follows:  for every record $(pos, val, i)$,
each party insert $(pos, val, i), i\in[1, len(val)]$ to a new set and get set $S_2^A, S_2^B$ for Alice and Bob. Then we compute
${\tt d2} = |S_2^A\cup S_2^B|$.


Finally we compute {\tt d = d2 - d1}.
%\section{Conclusion}

\appendix

\section*{Supplemental Information}
\section{{\sf ObliVM} Overview}

{\sf ObliVM} is a programming framework for secure computation, offering the 
following features:
\begin{itemize}
\item
{\bf Ease-of-use.} Non-specialist programmers can easily 
write programs in our source language {\sf ObliVM-lang}, a 
familiar imperative-style language.
\item
{\bf Efficiency.} {\sf ObliVM} compiles these programs into 
concise circuit representations suitable for secure computation. 
Much as MapReduce is a programming paradigm for 
parallel computation, {\sf ObliVM} offers user- and compiler- friendly
programming abstractions for secure computation.
\item
{\bf Formal security.}
{\sf ObliVM} offers a security type system to ensure 
that programs supplied by nonspecialist developers
will be executed securely without leaking information.
At a high level, the security type system guarantees
that a program's execution traces 
is oblivious to secret inputs.
\end{itemize}

\paragraph{Capabilities of {\sf ObliVM}.} 
{\sf ObliVM} supports richer and much more sophisticated applications
than the challenges given in this competition.
We have developed a variety of demo applications in machine learning,
streaming algorithms, graph algorithms, 
common data structures, and common utilities.
Using {\sf ObliVM}, we have demonstrated queries on {\sf GB} 
databases where we can compute the answers in a reasonable amount of time. 


%\paragraph{Whitepaper and more resources.}
\vspace{10pt}
For more information about {\sf ObliVM} and additional resources,
please refer to our whitepaper  

\begin{center}
\url{http://oblivm.com/papers/oblivm.pdf}
\end{center}

and our website 

\begin{center}
\url{http://oblivm.com}
\end{center}


\section{Thank You and Suggestions to the Organizers}
We think this competition is such a fantastic idea. 
It definitely helps to bridge communities, and helps 
state-of-the-art secure computation technology  
find its way in high-impact application domains.
We are grateful to the organizers for organizing this competition,
and for detailed discussions and feedback 
throughout.

We really hope that more competitions like this will be held in the future.
If a competition like this is to be repeated, we strongly 
suggest that the organizers increase the difficulty 
of the challenges --- e.g., consider much more sophisticated tasks
and bigger data sizes. 
We feel that the simple and small-scale nature of the problems  
in this competition is insufficient to fully demonstrate the true power
of state-of-the-art secure computation frameworks such as {\sf ObliVM}.


\bibliographystyle{plain}
\bibliography{refs,bibdiffpriv,bibliography,crypto,ref}


\end{document} 





